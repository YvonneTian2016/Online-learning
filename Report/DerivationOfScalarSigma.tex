\documentclass[twoside,11pt]{article}
\usepackage{amsmath,amsfonts,amssymb,amsthm}
\usepackage{graphicx,color}
\usepackage{verbatim,url}
\usepackage{listings}
\usepackage{upquote}
\usepackage[T1]{fontenc}
\usepackage[scaled]{beramono}
\usepackage{float}
\usepackage{parskip}
\usepackage{cite}

% Directories for other source files and images
\newcommand{\bibtexdir}{../bib}
\newcommand{\figdir}{eps}

\newcommand{\E}{\mathrm{E}}
\newcommand{\Var}{\mathrm{Var}}
\newcommand{\N}{\mathcal{N}}
\newcommand{\matlab}{{\sc Matlab}\ }

\setlength{\textheight}{9in} \setlength{\textwidth}{6.5in}
\setlength{\oddsidemargin}{-.25in}  % Centers text.
\setlength{\evensidemargin}{-.25in} %
\setlength{\topmargin}{0in} %
\setlength{\headheight}{0in} %
\setlength{\headsep}{0in} %

\renewcommand{\labelenumi}{(\alph{enumi})}
\renewcommand{\labelenumii}{(\arabic{enumii})}

\theoremstyle{definition}
\newtheorem{MatEx}{M{\scriptsize{ATLAB}} Usage Example}
\DeclareMathOperator*{\px}{\pmb{x}} 
\DeclareMathOperator*{\pu}{\pmb{u}} 
\DeclareMathOperator*{\rtr}{\mathrm{tr}} 

\definecolor{comments}{rgb}{0,.5,0}
\definecolor{backgnd}{rgb}{.95,.95,.95}
\definecolor{string}{rgb}{.2,.2,.2}
\lstset{language=Matlab}
\lstset{basicstyle=\small\ttfamily,
        mathescape=true,
        emptylines=1, showlines=true,
        backgroundcolor=\color{backgnd},
        commentstyle=\color{comments}\ttfamily, %\rmfamily,
        morecomment=[l]{\%},
        morecomment=[is]{\%\#}{\%\#},
        stringstyle=\color{string}\ttfamily,
        keywordstyle=\ttfamily, %\normalfont,
        showstringspaces=false}
\newcommand{\matp}{\mathbf{\gg}}

\newcommand{\myraise}{\vspace{-.15cm}}

\begin{document}
\pagestyle{headings}

\begin{center}
{\Large {\bf Derivation of Using Scalar Sigma in EM to update GMM}}  \\
\end{center}
\vspace{1cm}

\begin{enumerate} 
\item  \textbf {Derivation for Batch EM} \\
Assume the GMM composed of K Gaussian components, the pdf of the GMM$(\theta = (\pi_k,\mu_k,\Sigma_k)$) is:\\
\begin{equation}
p(x)=\sum_{k=1}^K\pi_kp(x|\mu_k,\Sigma_k)
\end{equation}
So the Likelihood Function of the GMM(N = the size of the dataset) should be:\\
\begin{equation}
\prod_{i=1}^Np(x_i)=\prod_{i=1}^N\{\sum_{k=1}^K\pi_kp(x_i|\mu_k,\Sigma_k)\}
\end{equation}
Because the possibility of each point is usually a very small number, and the production of many small numbers will cause the underflow issue with floating point numbers, we will \textbf{take its log} to transform the production function to a summation function and get the \textbf {log-likelihood function} as following:\par
\begin{equation}
\sum_{i=1}^N\log{\{\sum_{k=1}^K\pi_kp(x_i|\mu_k,\Sigma_k)\}}
\end{equation}
\textbf{Because the equation(1) p(x) will be expanded as following:}\\
\begin{equation}
p(x) = \sum_{k=1}^K\frac{\pi_k}{\sqrt{(2\pi)^d}|\Sigma_k|}e^{-\frac{1}{2}(x-\mu_k)^\top\Sigma_k^{-1}(x-\mu_k)}
\end{equation}
\textbf{And the $\Sigma_k $ is changed to be a scalar parameter:}  
\begin{align*}
\Sigma_k &= \sigma_k^2\\
&\Rightarrow |\Sigma_k|=(\sigma^2)^d \\
&\Rightarrow \Sigma_k^{-1}=\frac{1}{\sigma_k^2}\\
\end{align*}
Then the function becomes:\\
\begin{equation}
p(x) = \sum_{k=1}^m\frac{\pi_k}{\sqrt{(2\pi\sigma_k^2)^d}}e^{-\frac{1}{2}\frac{(x-\mu_k)^\top(x-\mu_k)}{\sigma_k^2}}
\end{equation}
\textbf{So the parameters in Gauss $\theta$ becomes}: $\theta = (\pi_k,\mu_k,\sigma_k^2)$  and we can use $h_k = \sigma_k^2$\\
makes $\theta = (\pi_k,\mu_k,h_k)$\par
Because I need to maximize the equation(3), however, the $\log\Sigma$ is a challenge when doing the maximization. The normal way to do that is to use \textbf{Jensen Inequality} as following: \\
So the log-likelihood function becomes (z is the latent variable):\\
$\sum_{i=1}^N\log{\sum_{k=1}^Kp(x_i)} = \sum_{i=1}^N\log\{\frac{\sum_{k=1}^Kp(x_i,z_i=k|\theta)}{p(z_i=k|\theta_t)}p(z_i=k|\theta_t)$\\
\begin{equation}
\ge \sum_{i=1}^N\sum_{k=1}^Kp(z_i=k|\theta_t)\log\{\frac{p(x_i,z_i=k|\theta)}{p(z_i=k|\theta_t)}\} 
\end{equation}

\textbf{Represent $\gamma_{ik}=p(z_i=k|\theta_t)$} ($\gamma_{ik} $ is the possibility that data $x_i$ belonged to the kth Gauss; $\theta_t $ is the updated parameter after the current E-step)\\
\textbf{Now I just need to maximize the equation (6) expanded as the following $L(\theta)$} \\
\begin{equation}
L(\theta)=\sum_{i=1}^N\sum_{k=1}^K\gamma_{jk}\left[\log{\pi_k}-\frac{d}{2}\log{2\pi h_k}-\frac{1}{2}\frac{(x_i-\mu_k)^\top(x_i-\mu_k)}{h_k}\right]
\end{equation}
And because we have a \textbf{constraint} that $\sum_{k=1}^K\pi_k = 1$, I use \textbf{Lagrange multiplier} to add this constraint in the maximization process as following $T(\theta)$.\\
\begin{equation}
T(\theta)= L(\theta)+\lambda(\sum_{k=1}^K\pi_k-1)
\end{equation}
\textbf{Maximization:}Get the derivation and set it to equal to zero.Then the parameters $\theta$ become:\\
$\frac{\partial T}{\partial \pi_k}=\sum_{i=1}^N\frac{\gamma_{ik}}{\pi_k}+\lambda = 0$\\
\begin{equation}
\Rightarrow \pi_k = \frac{\sum_{i=1}^N\gamma_{ik}}{\sum_{i=1}^N\sum_{k=1}^K\gamma_{ik}}
\end{equation}
$\frac{\partial T}{\partial\mu_k}=\sum_{i=1}^N\gamma_{ik}\frac{x_i-\mu_k}{h_k}=0$\\
\begin{equation}
\Rightarrow \mu_k = \frac{\sum_{i=1}^N\gamma_{ik}x_i}{\sum_{i=1}^N\gamma_{ik}}
\end{equation}
$\frac{\partial \pi}{\partial h_k}=-\sum_{i=1}^N\gamma_{ik}\frac{d}{2}\frac{1}{h_k}+\frac{1}{2}\sum_{i=1}^N\gamma_{ik}\frac{{(x_i-\mu_k)}^\top(x_i-\mu_k)}{h_k^2}=0$\\
\begin{equation}
\Rightarrow h_k= \frac{\sum_{i=1}^N\gamma_{ik}{(x_i-\mu_k)}^\top(x_i-\mu_k)}{d\sum_{i=1}^N\gamma_{ik}}
\end{equation}


\textbf {Finally, deviation result with }$\Sigma_k=h_kI$ is as following:\par
The two Steps in EM become:\par
\textbf {E-Step}:\\ 
$\gamma_ik=p(z^j=k|\theta^t)$\par

\textbf {M-Step}:\\
$\pi_k = \frac{\sum_{i=1}^N\gamma_{ik}}{\sum_{i=1}^N\sum_{k=1}^K\gamma_{ik}}$\par
$\mu_k = \frac{\sum_{i=1}^N\gamma_{ik}x_i}{\sum_{i=1}^N\gamma_{ik}}$\par
$h_k= \frac{\sum_{i=1}^N\gamma_{ik}{(x_i-\mu_k)}^\top(x_i-\mu_k)}{d\sum_{i=1}^N\gamma_{ik}}$\par
In this case, dimension d = 2 \par



\vspace{2cm}
\item \textbf {Derivation for Online EM$^{[1]}$}\\

 \textbf {Given the Gauvain's update formula for the matrix $\Sigma_j$ of j-th Gaussian in the mixture reads:}\par

\begin{equation}
\Sigma_j^{''} = \frac{b_jI+\Sigma_{q=0}^{N-1}\gamma_{qj}(s_q-\mu_j){(s_q-\mu_j)}^\top}{(a_j-2)+\Sigma_{q=0}^{N-1}\gamma_{qj}} 
\end{equation}
Because in the equation(12), scalars a,b and v are parameters of conjugate priors induced by the MAP solution$^{[2]}$. I is the identity matrix. a > d-1, b > 0 are hyper-parameters, and d = 2 is the dimension. Bishop[2006] provides details on the use of Dirichlet and Wishart distributions as conjugate priors.\par
So we use the same conjugate as the on-line paper$^{[2]}$ and by using the derivation result of $h_k$ as above, we can get the$ \Sigma_j $as following:\\
\begin{equation}
\Sigma_j^{'}  = \frac{b_jI+\Sigma_{q=0}^{N-1}\gamma_{qj}{(s_q-\mu_j)}^\top(s_q-\mu_j)}{(a_j-2)+d\Sigma_{q=0}^{N-1}\gamma_{qj}}
\end{equation}
 By using simple algebra, we get:
 \begin{equation}
 {(s_q-\mu_j)}^\top(s_q-\mu_j) = s_q^\top s_q - s_q^\top\mu_j-\mu_j^\top s_q+\mu_j\mu_j^\top
 \end{equation}
Substituting (14) into (13) and multiplying both the nominator and denominator by $\frac{1}{N}$, the formula for $\Sigma_j$ becomes:\\
\begin{align}
\Sigma_j^{'}  & = \Sigma_j^{'}  \frac{\frac{1}{N}}{\frac{1}{N}} \\
 & =\frac{\frac{b_j}{N}+\frac{1}{N}\Sigma_{q=0}^{N-1}\gamma_{qj}s_q^\top s_q-A+B}{\frac{(a_j-2)}{N}+\Sigma_{q=0}^{N-1}\frac{\gamma_{qj}}{N}} \\
 \text{where,~~~~~~~~}  A  &= \frac{1}{N}\Sigma_{q=0}^{N-1}\gamma_{qj}s_q^\top\mu_j+\frac{1}{N}\Sigma_{q=0}^{N-1}\gamma_{qj}\mu_j^\top s_q  \\
B & = \frac{1}{N}\Sigma_{q=0}^{N-1}\gamma_{qj}\mu_j^\top\mu_j  
\end{align}


Then the fact to obtain MAP update formula$^{[1]}$ of the covariance matrix(the sufficient statistics: a triplet $u_i^j=((u_\gamma)_i^j,(s)_i^j,(ss^\top)_i^j)$)is:\\
\begin{align}
\Sigma_j^{'} & = \frac{\frac{b_jI}{N}+{(s^\top s)}_i^j-A+{(u_\gamma)}_i^jB}{\frac{(a_j-2)}{N}+d(u_\gamma)_i^j}\\
 \text{where,~~~~~~~~} A &={(s^\top)}_i^j\mu_j+\mu_j^\top s_i^j \\
B &=\mu_j^\top\mu_j \\
\end{align}
 
This is a article ralted to Bibtex\cite{viola2001rapid}\\
The second citition\cite{murphy2012machine}


\end{enumerate}
\vspace{3cm}
{\Large\bf{Refrence} }\par
[1] Vorba, Jiri, et al. "On-line Learning of Parametric Mixture Models for Light Transport Simulation?Supplemental material." \par
[2] Vorba, Ji?�, et al. "On-line learning of parametric mixture models for light transport simulation." ACM Transactions on Graphics (TOG) 33.4 (2014): 101.\par
[3] Bishop, Christopher M. "Pattern Recognition." Machine Learning (2006).\par

\bibliographystyle{unsrt} %
\bibliography{bibfile}

\end{document}